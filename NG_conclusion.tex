\chapter{Conclusions and Future Work}
\label{ch:conclusion}

The goal of this thesis was to explore a viable and affordable method of incorporating geothermal energy as a prime power source to form a microgrid using an organic Rankine system. Low temperature geothermal sites were looked at specifically because such sites can be found within Alaska and similar environments which are not currently fully utilized. To achieve the goal, a model was developed to simulate the thermodynamic and electrical processes of the ORC and the SEIG. Next, two different systems were examined: a greenfield site with minimal existing infrastructure and a brownfield site that is already connected to an electric grid for pump power. 

\section{Model Validation}
In order to verify that the model approximates reality, a Electratherm Green Machine was simulated under testing conditions. These values were compared to the results of a Green Machine test at the University of Alaska Fairbanks. There were some difficulties in re-creating exact conditions because some values were not precisely recorded. For example, operating ranges for the high and low pressures of the working fluid were stated, but specific measurements were not included. 
Other values, were assumed while sizing the ORC for the test but not measured after the installation because the report was interested in the performance of the whole unit rather than modeling it. These values include the heat transfer coefficients and areas of the heat exchangers, as well as the isentropic efficiencies of the pump and expander.
% area/U discrpencies

Despite the differences in these parameters between the model and the report, the model was still validated based on the general trends in the output power. Higher source fluid temperatures yield more power. Greater mass flow rates of the source and sink similarly provide a greater output power, though this is limited by the parameters of the heat exchangers. Additionally, it was shown that a larger heat exchanger area will transfer heat at a greater rate than a smaller area, while accounting for mass flow rates. These patterns and trends follow what is expected for an organic Rankine cycle system.

\section{Greenfield}
%It was shown that output of an ORC could produce a net output of around \SIrange{27.5}{30.8}{\kilo\watt} over the course of the year. This is due to a steady temperature hot water resource and a cold water sink that varies over seasons but remains liquid. 
The greenfield simulations demonstrated the expected increase of available power output during winter months. 
According to the model, even more power could be produced by increasing the mass flow rate of the working fluid. However, this assumes the isentropic efficiency of the turbine expander is equal in the two cases, which is unlikely. The additional mass means the fluid does not fully vaporize, which is not typical for ORC systems because the efficiency is not constant under all conditions. 

While the available power gained by increasing the flow rate of the working fluid is limited by the vaporization, that flow rate limit could be increased by also increasing the flow rates of the source and sink fluids. Hypothetically, the higher flow rates mean larger pumps and heat exchangers must be used for the system, increasing the total cost, but that should be offset by the increase in output power. Realistically, though, only so much hot water can be drawn from geothermal resources before the source temperature begins to drop. This can be alleviated by re-injecting the source effluent up to a point. However, the model assumes the resources are steady, and if the flow rates are increased too much, that assumption would no longer be valid.


\section{Brownfield}
In the brownfield case study it was shown the ORC could not produce enough electrical power to drive the district heating loop pumps from the inverter as a prime power source.
%The expected net power produced by an ORC for a brownfield system is \SI{27.8}{\kilo\watt}, just under the desired \SI{30}{\kilo\watt} needed to drive the district heating pumps.
However, there was sufficient power produced, but due to inefficiencies and parasitic loads there was not enough remaining to power everything. 
%The gross electrical power produced by the generator before inverter losses is just shy of \SI{30.8}{\kilo\watt}. 
This indicates there may be enough power from the local resource to operate the district heating system while the electrical grid is active and capable of regulating the frequency and voltage of the induction generator. However, the whole system would still go down during a power outage and the diesel generators would still need to be used to give the community heat, albeit with lower fuel consumption than without the ORC system.

One possible solution is to increase the size of the evaporator area. This could allow the more heat to flow from the water to the working fluid, meaning the more fluid could be moved while still fully vaporizing. Unfortunately, the cost of the system would go up as well due to larger heat exchanger.

%check the mech power produced
Another option could be to bypass electrical conversion entirely, eliminating several inefficiencies along the way. The simulated ORC produced more mechanical power than electrical, and the mechanical load to drive the pumps is actually less than the rated electrical load.
%The simulated ORC produced about \SI{31.7}{\kilo\watt} of mechanical power, and the \SI{30}{\kilo\watt} district heating pumps actually need slightly less mechanical power to operate.
A system could be developed to connect the ORC expander to a mechanical pump to drive the district heating loop. Depending on the design, this could be done with a direct coupling or through a gearbox of some kind. Of course, other inefficiencies would be introduced and controlling the flow would be challenging, but it might be worth further examination.

\section{Future Work}
One improvement that the model would need before it can be more widely is a user friendly graphical interface. Currently, an MATLAB script is run in order to initialize all the input variables and parameters, then the Simulink simulation can be run, and finally a second script is run to generate the plots. Combining some of these steps into a graphical user interface, this model could be more readily utilized by others. In addition to the interface, there are other potential improvements relating the thermal, electrical, and transient aspects of the model.

\subsection{Thermal Model}
As described previously, 
%more heat was transferred through the heat exchangers in the model than what was measured in actual testing of an ORC device. A possible explanation is that 
the model is perfectly efficient at transferring the heat and none is lost to the ambient environment. In reality, fluid in the heat exchangers, pumps, expanders, and pipes will experience some amount of heat loss or gain to the surrounding area. Furthermore, the model only calculated pressure changes across the pump and expander, where as fluids in actual systems will see some pressure drops through each component. The overall accuracy of the model could be improved by accounting for this additional routes of heat flow and pressure changes.

Another implementation that could improve system performance is an optional pre-heater that several commercial systems already use. This heat exchanger is inserted in the loop between the pump and evaporator on the high pressure side effectively increasing the area of the evaporator. Sometimes the pre-heater uses a different heat source from the primary source. Even the heat of the working fluid coming off of the expander before flowing through the condenser could be used. This allows the system to recycle some of the heat that was not converted into mechanical energy letting it operate at a slightly higher efficiency at the cost of an additional component.

\subsection{Electrical Model}
The microgrids simulated as a part of this project only included squirrel cage induction generators because they are relatively inexpensive to build and maintain, and are therefore commonly used in commercial ORC systems. They are not, however, ubiquitous. It would be beneficial to include models of additional generator types such as DC or permanent magnet synchronous machines. This would allow the user to compare systems made by different manufacturers which use various generator technologies.
%various inverter models
%add storage

\subsection{Transient Model}
This model was designed to examine steady-state results. It looks at the energy balance of each component to simulate how they interact over long term. This means it is unable to accurately capture transient dynamics after a change in load, temperature, or flow rate. Each of those fluctuations can occur at different time scales and effect the system dynamics differently. It is important to know the system is capable of providing primary power while maintaining grid frequency and voltage under those different perturbations.

\section{Final Thoughts}
Overall, this thesis demonstrated that for remote greenfield microgrids, modest loads can be primarily powered off of these low temperature geothermal organic Rankine cycles. At the Pilgrim hot springs in Alaska, a small greenhouse could be designed and built to operate in the winter using LED grow lights. This could provide a source of fresh local vegetables to the nearby community of Nome all year. 

For brownfield microgrids, there is less flexibility because the loads are already present and cannot be designed around the available source. For the village of Bergsta$\eth$ir, Iceland the geothermal ORC prime power system presented here could not, on its own, fully supply the existing district heating system. The ORC could, however, provide most of the necessary power, meaning the a significant portion of the diesel fuel used to operate the district heating system during power outages could be displaced.

\cleardoublepage


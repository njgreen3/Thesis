\begin{abstract}

Diesel electric generation is heavily used in remote permanently islanded microgrids, even in areas where alternative resources are readily available. The cost of diesel fuel for these generators is high in part because of the high cost of transportation to these distant sites using vehicles that rely on some form of petroleum themselves. 
%Additionally, as the cost of fuel increases, so too does the cost of transportation, hitting these remote communities harder. 
This thesis will show that local resources such as geothermal hot springs can provide primary power for these remote microgrids, even at relatively low temperatures below the boiling point of water. The geothermal heat will be converted to electrical energy using an organic Rankine cycle turbine in combination with a self-excited induction generator. A steady-state energy balance model will be developed using MATLAB Simulink to simulate greenfield and brownfield geothermal microgrids at Pilgrim Hot Springs, Alaska and Bergsta$\eth$ir, Iceland, respectively, to demonstrate viability of this microgrid design. 
The results of the simulations have shown modest loads can be primarily powered off of these low temperature geothermal organic Rankine cycles over long time scales. As expected, more power is available during colder months when sink temperatures are lower. More research should be done to examine system response over shorter time scale transients, which are not covered in the scope of this work.

%A stability analysis will be conducted using the long and short term simulation models which evaluate system voltage and frequency under different conversion topologies. A cost analysis will also be conducted to compare the economic viability of operating the different systems in remote communities. It is expected the system with greatest stability will not be the most cost-effective, but that there is a system which provides stable power within reasonable tolerances at optimal cost.

\end{abstract}

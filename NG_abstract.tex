\begin{abstract}

Diesel electric generation is heavily used in remote permanently islanded microgrids, even in areas where alternative resources are readily available. The cost of diesel fuel for these generators is high in part because of the difficulty in transportation. Additionally, as the cost of fuel increases, so too does the cost of transportation, hitting these remote communities harder. This thesis will show that local resources such as geothermal hot springs can provide primary for these remote microgrids, even at relatively low temperatures below the boiling point of water. The geothermal heat will be processed with organic Rankine cycle in combination with a self excited induction generator. A steady-state energy balance model will be devoloped using MATLAB Simulink. The model will be used to simulate greenfield and brownfield geothermal microgrids at Pilgrim Hot Springs, Alaska and Bergssta$\eth$ir, Iceland respectively to demonstrate viability of this microgrid design. 

%A stability analysis will be conducted using the long and short term simulation models which evaluate system voltage and frequency under different conversion topologies. A cost analysis will also be conducted to compare the economic viability of operating the different systems in remote communities. It is expected the system with greatest stability will not be the most cost-effective, but that there is a system which provides stable power within reasonable tolerances at optimal cost.

\end{abstract}

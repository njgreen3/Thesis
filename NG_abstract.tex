\begin{abstract}

Power stability regulation and high installation costs are major barriers to adoption of renewable energy resources in permanently islanded remote microgrids. In general it will be shown for these remote microgrids which combinations of bus architectures and converter topologies provide optimal system stability and cost. This will be achieved by simulating a hypothetical geothermal-diesel hybrid microgrid at Pilgrim Hot Springs using various conversion topologies. The systems will be modelled from a short time dynamic stability perspective using PSpice and/or MATLAB and longer-term average energy standpoint using MATLAB. A stability analysis will be conducted using the long and short term simulation models which evaluate system voltage and frequency under different conversion topologies. A cost analysis will also be conducted to compare the economic viability of operating the different systems in remote communities. It is expected the system with greatest stability will not be the most cost-effective, but that there is a system which provides stable power within reasonable tolerances at optimal cost.

\end{abstract}

\chapter{Analysis and Results}
\label{ch:analysis}

With the model constructed as described in \autoref{ch:model}, simulations can be run under the different scenarios: Greenfield and Brownfield. For the former, no electrical grid is currently present, while for the later there is existing electrical infrastructure.

\section{Greenfield Scenario --- Alaska}
This system is located approximately two hours north of Nome, Alaska on the western coast of the state. The hot water resource is drawn from the Pilgrim Hot Springs. The hot water is assumed to be drawn at a temperature of $\SI{81.3}{\degreeCelsius}$ and a rate of $\SI{15.2}{\liter\per\second}$, while the cooler water is drawn at a temperature of $\SI{8.5}{\degreeCelsius}$ at $\SI{15.6}{\liter\per\second}$ to provide a heat sink \cite{Haselwimmer2013}. It is assumed the water is drawn in at atmospheric pressure.

The working fluid is R245-fa, a typical refrigerant used by ORC systems. The operating high and low pressure set points of the working fluid are $\SI{1e6}{\pascal}$ and $\SI{1.5e5}{\pascal}$. The evaporator has a heat transfer area of $\SI{20.3}{\meter\squared}$, and an effective heat transfer coefficient of $\SI{1500}{\watt\per\kelvin\per\meter\squared}$. The condenser has the same area, but an effective heat transfer coefficient of $\SI{1400}{\watt\per\kelvin\per\meter\squared}$.

\section{Brownfield Scenario --- Iceland}
Bergssta$\eth$ir, Iceland is located three to four hours from Reykjavík on the norther coast of the country. The hot water resource is drawn from the Pilgrim Hot Springs.

\chapter{Geothermal Energy}
\label{ch:geothermal}

\section{High and Low Enthalpy Sources}
Enthalpy is the energy of a system available to be converted to work. It is related to the temperature of the geothermal resource

\section{Geothermal Cycles}
Describe each of the following but focus on cycles for low enthalpy sources
\begin{description}
\item[Dry Steam] High enthalpy - send hot steam directly through turbine then condensed and injected back underground
\item[Flash Steam] High enthalpy - high pressure hot water is allowed to boil becoming steam and low pressure hot water. Steam is sent through turbine then condensed, recombined with water, and injected back underground
\item[Binary Cycle] Technically not limited to low enthalpy. Two loops, heat source loop and working fluid loop. Can be divided into
\begin{description}
\item[(Organic) Rankine Cycle]
\item[Kalina Cycle]
\end{description}
\end{description}

\section{Synchronous and Asynchronous Generators}
This section should focus the difference between the two with respect to geothermal sources. I can have a fundemental description of the generators either in this chapter or the intro chapter.


\section{Pilgrim Hot Springs}
More thourough description of the resource and potential development plans.



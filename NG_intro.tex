\chapter{Introduction}

\section{Problem Statement}
The state of Alaska currently has dozens of communities which are electrically isolated from the rest of the state. They effectively act as remote permanently islanded microgrids. These communities typically use diesel generators to provide most of their electrical power. Power sources such as coal or natural gas are less expensive in larger grids but these microgrids are too small to take advantage of such economies of scale. This makes it expensive to operate relative to the size of the communities.

One method of reducing operating costs is to offset diesel fuel through the usage of renewable energy resources.


\section{Microgrid Description}
Microgrids are electrical systems composed of sources and loads within a well defined electrical boundary. Energy storage is often incorporated into systems as well, but not necessarily. Many microgrids are connected to a larger grid but with the ability to become separated, or islanded, while still maintaining some or all of the loads.

A microgrid can operate using Alternating Current (AC), Direct Current (DC), or a hybridization of the two. The choise to whether to use AC or DC depends on the demands of the loads, the available power sources and conversion devices, as well as the existing electrical infrastructure.

Approximately 70\% of the population of Alaska is connected to a single grid called the Railbelt \cite{railbelt}. The Railbelt stretches over 600 miles from the Homer on the Kenai Peninsula to the greater Fairbanks area in the interior and includes most communities on the road system in between. The remaining communities and villages are effectively permanently islanded microgrids. 

\subsection{Sources}
Microgrid power is typically generated by Distributed Energy Resources (DER). This can include renewable resources such as solar photovoltaics, wind turbine generators, and geothermal as well as and non-renewable sources such diesel generators and natural gas microturbines. 

Diesel generators use the combustion of diesel fuel to spin an alternator producing AC power. These generators are prolific in rural Alaska generally being used as the prime mover to regulate grid frequency and voltage. 
%These generators are composed of an engine, alternator, fuel system, excitation system, governor, cooling system, exhaust system, and turbocharger.
Due to their widespread use, significant energy cost savings can be generated by reducing fuel consumption through efficiency improvements and use of alternate energy sources.

Geothermal generators are heat engines and can operate similarly to traditional coal and nuclear plants. Heat causes a working fluid to thermally expand and change phases thus spinning a turbine to produce AC power. The primary difference is that geothermal systems get heat from the Earth as opposed to combustion or nuclear reactions. Geothermal generators can be divided into high heat and low heat categories. High heat geothermal systems work with temperatures at and above the boiling point of water which means water can be used as the working fluid. Low heat systems operate at temperatures below the boiling point of water therefore must use a refrigerant as an alternative working fluid. 

Wind turbine generators 
%Describe distributed energy resources (DER) in general, diesel gensets, geothermal sources, and 
\begin{verbatim} 
 Address wind & solar too.
\end{verbatim}

\subsection{Loads}
Well designed microgrids are designed around expected loads. 
\begin{verbatim} 
 Describe dispatchable loads and give some examples. 
\end{verbatim}

\subsection{Storage}
Energy storage is beneficial to microgrid operation because it allows generation to be spread over time. Such devices can include batteries, flywheels, supercapacitors, pumped hydro, and more. These devices have different storage duration and discharge times making different storage technologies advantageous in different situations \cite{Schoenung2003}. The discharge of bulk energy storage over the course of many hours allows for load leveling and provides spinning reserve for the grid. Load peak shaving typically involves a discharge time from minutes to several hours. Energy storage discharge over seconds and subseconds is generally done to improve power quality.
\begin{verbatim}
 Define spinning reserve somewhere
\end{verbatim}

\subsection{Control}
A microgrid control system ties all the other components together. Control systems monitor maintain voltage and frequency of the grid while ensuring sufficient active and reactive power is supplied to the loads.

\subsection{Conversion}
Conversion devices take a form of electrical power (AC or DC) and convert it into a different form. Inverters convert DC power into sinusoidal AC power. Rectifiers convert AC to DC. DC-DC converters can step up or down the voltage level of a DC power source. Transformers can step up or down the voltage level of an AC source while maintaining frequency. Modifying the frequency of an AC source typically involves rectifying the source then inverting then that output at the desired frequency.

